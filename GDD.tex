\documentclass[a4paper, 11pt]{article} 
\usepackage[utf8]{inputenc} 
\usepackage{indentfirst} 
 
\title{\textbf{Sohlepse} \\ \small{Um Game Design Document}} 
\author{Larissa Goto Sala, Pedro Paulo de Queiroz Bambace, Nathalia Orlandi Borin} 
 
\begin{document} 
 
\maketitle 
 
\newpage 
 
\newpage 
 
\section{Resumo} 
	 
	Sohlepse (nome passível de mudanças) é um jogo de puzzle-plataforma que faz uso de reflexões para proporcionar o controle de mais de um personagem com os mesmos comandos. 
 
Isso pode ser feito de forma paralela (os dois personagens se movem ao mesmo tempo) ou sequencial (a mesma sequência de movimentos de um personagem é repetida enquanto você move o outro personagem). 
 
Cada fase apresentará diferentes obstáculos que te obrigará a fazer uso criativo de suas reflexões para superá-los. 
 
\section{Jogabilidade} 
 
	\subsection{Ações do jogador} 
		\begin{itemize} 
			\item Andar: $\rightarrow$ e $\leftarrow$ ou $A$ e $D$  
			\item Pular: $\uparrow$ ou $W$
			\item Empurrar caixas: $E$ enquanto anda na direção da caixa  
			\item Pressionar botões: permanecer em cima dele 
			\item Acionar alavancas: $E$
		\end{itemize} 
 
	\subsection{Objetos do jogo} 
		\begin{itemize} 
			\item Botões: ativam ou desativam uma quantidade arbitrária de paredes e espinhos divididos em 2 tipos: 
			\begin{itemize}
			        \item Os que se mantêm ativos enquanto estão pressionados
			        \item Os que pressionados uma vez, se mantém ativos até o fim da fase
			\end{itemize}   
			\item Alavancas - Se mantêm ativos até serem acionadas novamente - ativam ou desativam uma quantidade arbitrária de paredes e espinhos 
			\item Caixas - Podem ser empurradas para formar escadas ou acionar botões 
			\item Chão quebrável - Se quebram quando o jogador passa por ele duas vezes 
			\item Água - Jogador pode passar por esse tile normalmente, mas sua movimentação fica mais lenta 
			\item Lava - Jogador morre se encostar nesse tile 
			\item Espetos - Jogador morre se encostar nesse tile - pode ser desativado por um botão/alavanca 
		\end{itemize} 
 		\subsection{EndGoal}
 			Espelho que representa o ponto onde o Jogador deve chegar para terminar a fase, ao interagir com ele, a próxima fase começa(dependendo da fase podem haver cutscenes antes da inicialização)
	
\section{Organização das cenas} 
		\subsection{Menu Principal} 
			\begin{itemize} 
				\item Jogar - Carrega o última fase que o jogador ainda não completou,armazenado na variável global unlocked\_stage. 
				\item Seleção de fases - Carrega o menu de seleção de fases. Suporta apenas 12 fases atualmente, precisa ser melhorado.
				\item Opções - Carrega o menu de opções, que não existe ainda, precisa ser feito. 
				\item Créditos - Carrega os créditos 
				\item Sair - Fecha o jogo.
			\end{itemize} 
 
		\subsection{Stage Manager} 
			Faz o trabalho de carregar a fase atual nos dois viewports, assim como organiza as telas horizontal ou verticalmente, como definido na variável MODE.
			\begin{enumerate}
			\item Setup - Configura as câmeras de cada viewport para seguir cada player, de acordo com a variável PLAYER, que está em Stage. Também limita as câmeras aos retângulos definidos como Real e Mirrored.
				\begin{enumerate}
				\item Viewports - Contidos em containers verticais ou horizontais, o Setup define qual será usado.
				\item Stage - Apresenta as variáveis ACT, PLAYERS, MAX\_CLONES, e MODE (1 é vertical, 2 é horizontal).
				\end{enumerate}
			\item Pause Menu
				\begin{enumerate}
				\item Voltar ao jogo - Retoma o jogo.
				\item Menu - Volta ao menu principal
				\item Recomeçar - Reinicia a fase.
				\item Controles - Precisa ser feito.
				\item Sair - Fecha o jogo, mas pede confirmação primeiro.
				\end{enumerate}
			\end{enumerate}

		\subsection{Stage}
			Nó raiz de cada fase.
			\begin{itemize}
				\item TileMap - Layout da fase.
				\item Real e Mirrored - ReferenceRect, definem os limites da câmera.
				\item Endgoal e EndgoalMirror - Objetivo final de cada fase.
				\item Players - Um Node normal, cujos filhos são instâncias do Player.
				\item Reality e Mirror - dois Nodes normais, cujos filhos são os nós Objects e Hazardz nas fases Real e Espelhada respectivamente
				\item Objects - Um Node normal, cujos filhos são botões, paredes, alavancas, etc.
				\item Hazards - Um Node normal, cujos filhos são limites, firearea, espetos, etc.
			\end{itemize}

		\subsection{Player}
			O Player é o personagem humano controlado pelo jogador, que através da movimentação e interação com os objetos da fase, deve chegar no EndGoal, evitando cair e  tocar em Hazards

		\subsection{Botões e paredes}
			Nesta seção, paredes se referem à Walls, MovingPlatforms e Thorns. E botões se refere à Button, ButtonInst ou Levers.
			\begin{itemize}
				\item 1 parede -> 1 botão:
				\\
				Coloque a parede num grupo com o nome do botão.
				\item 1 parede -> 2-3 botões (alternado):
				\\
				Coloque a parede em 2-3 grupos, cada um com o nome do botão desejado.
				A parede deve ter activation = 0.
				\item 1 acionável -> 2-3 ações (cumulativas):
				\\
				Coloque a parede em 2-3 grupos, cada um com o nome do botão desejado.
				A parede deve ter activation = 1.
				\item 2-3 paredes -> 1 botão
				\\
				Coloque as 2-3 paredes no grupo do botão.
				\item Parede 1 -> Botão A e B (alternado), Parede 2 -> Botão B
				\\
				Coloque a Parede 1 nos grupos A e B, e a Parede 2 no grupo B.
				A parede 1 deve ter activation = 0.
				\item Parede 1 -> Botão A e B (cumulativo), Parede 2 -> Botão B
				\\
				Coloque a Parede 1 nos grupos A e B, assim como os próprios botões, e a Parede 2 no grupo B. 
				A parede 1 deve ter activation = 1.
				\item Parede 1 -> Botão A, B e C (cumulativo), Parede 2 -> Botão B e C (alternado), Parede 3 -> Botão A
				\\
				Coloque a Parede 1 nos grupos A.B e C. Coloque a Parede 2 nos grupos B e C. Coloque a Parede 3 no grupo A. Coloque os botões A,B e C em seus respectivos grupos.
				A parede 1 deve ter activation = 1, e a parede 2 deve ter activation = 0.
				\item Parede 1 -> Botão A, B e C (cumulativo), Parede 2 -> Botão B e C (alternado), Parede 3 -> Botão A
				\\
				Coloque a Parede 1 nos grupos A.B e C. Coloque a Parede 2 nos grupos B e C. Coloque a Parede 3 no grupo A. Coloque os botões A,B e C em seus respectivos grupos.
				As paredes 1 e 2 devem ter activation = 1
				\item CASO ESPECIAL: \\
				Para facilitar a organização, em botões do tipo ButtonInst que \textbf{ativam mais de dois grupos de paredes de forma acumulada}, você pode colocar o nome dos grupos na array \textit{names}, inserir o ButtonInst nesses dois grupos, e colocar as paredes acionadas por eles em seu respectivo grupo.
				Ex:
				Parede 1 -> Botão A e C (alternada)
				Parede 2 -> Botão A e B (acumulada) 
 				Parede 3 -> Botão B e C (alternada)
 				Names(A) = ["A", "A2"], Names(B) = ["B", "B2"]
 				Botão A está nos grupos A e A2, Botão B está nos grupos B e B2, Botão C está no grupo C.
 				Parede 1 está nos grupos A e C, Parede 2 está no grupos A2 e B2, Parede 3 está nos grupos B e C.

\section{Ato I} 
 
	Duas telas espelhadas, podendo a divisão ser na horizontal ou vertical, dependendo da fase. O movimento do 
	personagem na primeira tela é refletido de maneira idêntica na segunda tela. Uma das telas pode ser obscurecida 
	e/ou conter partes quebradas/cobertas dependendo da fase. 
	Nesse ato, o principal objetivo é introduzir o jogador às mecânicas do jogo. As fases são idênticas 
	em ambos lados do espelho (salvo possíveis exceções onde o espelho será obscurecido/quebrado). 
 
\subsection{Mecânicas} 
 
	Lista das mecânicas do ato 
	\begin{itemize} 
		\item Algo 
		\item Algo 
	\end{itemize} 
 
\subsection{Possíveis fases} 
 
	Descrições/desenhos mais pra frente 
	\begin{itemize} 
		\item Fase1 : Tutorial de movimento, interação e pausa 
		\item Fase2 : Introduz caixas para subir e alavancas
		\item Fase3 : Introduz água, lava e espinhos
		\item Fase4 : Introduz caixas para apertar botões
		\item Fase5 : Introduz espinhos acionáveis e plataformas one-way
		\item Fase6 : Introduz falling floor e plataformas que se movem
		\item Fase7 : ?
		\item Fase8 : ?
		\item Fase9 : Introduz mecânica de plataforma que empurra caixa, morte por esmagamento
		\item Fase10 : Tutorial de restart, introduz alavancas cumulativas e que ativam mais de uma coisa (pequenas diferenças entre as paredes)
		\item Fase11 : Uso de falling floors, alavancas que ativam mais de uma coisa - patience platforming (pequenas diferenças entre água e lava)
		\item Fase12 : Fase final do Ato 1 - usa da dessincronização.
	\end{itemize} 
 
\subsection{História} 
 
	Jogador acorda em uma sala fechada, com apenas um espelho, sem saber onde está. Ao interagir com o espelho, é puxado pela própria reflexão para o "mundo espelhado".
	Percorre o laboratório e vai percebendo sutis mudanças na reflexão das fases. Eventualmente, encontra saída, mas é quase morto pela própria reflexão.

\section{Ato II} 
 
	Duas telas que representam fases diferentes, podendo a divisão ser na horizontal ou vertical, dependendo da fase. 
	O movimento do personagem na primeira tela é refletido na segunda tela, mas como os cenários são diferentes, pode 
	haver uma dessincronização entre os movimentos 
	Nesse ato, o mais longo do jogo, o objetivo é brincar com a noção de realidades paralelas, onde você deve planejar 
	bem seus movimentos para controlar dois personagens em cenários distintos com os mesmos inputs. Ao longo das fases, 
	a barreira entre as duas realidades começa a se quebrar, e você pode acabar interagindo com sua "reflexão" de  
	maneiras diferentes. 
 
	Lista das mecânicas do ato 
	\begin{itemize} 
		\item Algo 
		\item Algo 
	\end{itemize} 
 
\subsection{Possíveis fases} 
 
	Descrições/desenhos mais pra frente 
	\begin{itemize} 
		\item Fase13 : ?
		\item Fase14 : ?
	\end{itemize} 
 
\subsection{História} 
 
	Resumo do que acontece 
 
\section{Ato III} 
 
	Nesse ato, a realidade do espelho se fundiu quase completamente com a realidade, então as fases se passam numa única 
	tela. O diferencial desse ato é o uso de "clones", cujo número pode ser limitado ou não, dependendo da fase, que repetem o que você faz, como numa gravação. A ideia é sincronizar o seu movimento com o dos seus "clones" para conseguir atingir seu objetivo. 
 
	Lista das mecânicas do ato 
	\begin{itemize} 
		\item Algo 
		\item Algo 
	\end{itemize} 
 
\subsection{Possíveis fases} 
 
	Descrições/desenhos mais pra frente 
	\begin{itemize} 
		\item Fase 
		\item Fase 
	\end{itemize} 
 
\subsection{História} 
 
	Resumo do que acontece 
 
\section{Assets} 
 
\subsection{Placeholders} 
	Em progresso 
 
\subsection{Definitivos} 
	Em progresso 
 
\end{document} 
 