\documentclass[a4paper, 11pt]{article}
\usepackage[utf8]{inputenc}
\usepackage{indentfirst}

\title{\textbf{Sohlepse} \\ \small{Um Game Design Document}}
\author{Larissa Goto Sala, Pedro Paulo Bambace de Queiroz, Nathalia Orlandi Borin}

\begin{document}

\maketitle

\newpage

\newpage

\section{Resumo}
	
	Insira um resumo aqui

\section{Jogabilidade}

	\subsection{Ações do jogador}
		\begin{itemize}
			\item Andar 
			\item Pular 
			\item Empurrar caixas 
			\item Pressionar botões
			\item Acionar alavancas 
		\end{itemize}

	\subsection{Objetos do jogo}
		\begin{itemize}
			\item Botões - se mantém ativos enquanto estão pressionados - abrem ou fecham portas
			\item Alavancas - se mantém ativos até serem acionadas novamente - abrem ou fecham portas
			\item Caixas - podem ser empurradas para formar escadas ou acionar botões
			\item Chão quebrável - se quebram quando o jogador passa por ele duas vezes
			\item Água - jogador pode passar por esse tile normalmente
			\item Lava - jogador morre se encostar nesse tile
		\end{itemize}


\section{Ato I}

	Duas telas espelhadas, podendo a divisão ser na horizontal ou vertical, dependendo da fase. O movimento do
	personagem na primeira tela é refletido de maneira idêntica na segunda tela. Uma das telas pode ser obscurecida
	e/ou conter partes quebradas/cobertas dependendo da fase.
	Nesse ato, o principal objetivo é introduzir o jogador às mecânicas do jogo. As fases são idênticas
	em ambos lados do espelho (salvo possíveis exceções onde o espelho será obscurecido/quebrado).

\subsection{Mecânicas}

	Lista das mecânicas do ato
	\begin{itemize}
		\item Algo
		\item Algo
	\end{itemize}

\subsection{Possíveis fases}

	Descrições/desenhos mais pra frente
	\begin{itemize}
		\item Fase
		\item Fase
	\end{itemize}

\subsection{História}

	Resumo do que acontece 

\section{Ato II}

	Duas telas que representam fases diferentes, podendo a divisão ser na horizontal ou vertical, dependendo da fase.
	O movimento do personagem na primeira tela é refletido na segunda tela, mas como os cenários são diferentes, pode
	haver uma dessincronização entre os movimentos
	Nesse ato, o mais longo do jogo, o objetivo é brincar com a noção de realidades paralelas, onde você deve planejar
	bem seus movimentos para controlar dois personagens em cenários distintos com os mesmos inputs. Ao longo das fases,
	a barreira entre as duas realidades começa a se quebrar, e você pode acabar interagindo com sua "reflexão" de 
	maneiras diferentes.

	Lista das mecânicas do ato
	\begin{itemize}
		\item Algo
		\item Algo
	\end{itemize}

\subsection{Possíveis fases}

	Descrições/desenhos mais pra frente
	\begin{itemize}
		\item Fase
		\item Fase
	\end{itemize}

\subsection{História}

	Resumo do que acontece

\section{Ato III}

	Nesse ato, a realidade do espelho se fundiu quase completamente com a realidade, então as fases se passam numa única
	tela. O diferencial desse ato é o uso de "clones", cujo número pode ser limitado ou não, dependendo da fase, que repetem o que você faz, como numa gravação. A ideia é sincronizar o seu movimento com o dos seus "clones" para conseguir atingir seu objetivo.

	Lista das mecânicas do ato
	\begin{itemize}
		\item Algo
		\item Algo
	\end{itemize}

\subsection{Possíveis fases}

	Descrições/desenhos mais pra frente
	\begin{itemize}
		\item Fase
		\item Fase
	\end{itemize}

\subsection{História}

	Resumo do que acontece

\section{Assets}

\subsection{Placeholders}
	Em progresso

\subsection{Definitivos}
	Em progresso

\end{document}
