\documentclass[a4paper, 11pt]{article}
\usepackage[utf8]{inputenc}
\usepackage{indentfirst}
\usepackage{hyperref}

\title{\textbf{MAC0214 - Atividade Curricular em Cultura e Extensão} \\ \small{Segundo semestre de 2018}}
\author{Larissa Goto Sala, Pedro Paulo Bambace de Queiroz, Nathalia Orlandi Borin}

\begin{document}

\maketitle

\newpage

\newpage

\section{Introdução}
	
	Nosso projeto será um jogo desenvolvido do início ao fim como parte das atividades da USPGameDev
	desse semestre.

\section{Sobre o jogo}
\subsection{Resumo}

	Sohlepse (nome passível de mudanças) é um jogo de \textit{puzzle}-plataforma que faz uso
 de reflexões para proporcionar o controle de mais de um personagem com os mesmos comandos.

 	Isso pode ser feito de forma paralela (os dois personagens se movem ao mesmo tempo) ou
 sequencial (a mesma sequência de movimentos de um personagem é repetida enquanto você
 move o outro personagem).

 	Cada fase apresentará diferentes obstáculos que te obrigará a fazer uso criativo de suas
 reflexões para superá-los.

\subsection{Estrutura}
	Usaremos a \textit{engine open source Godot}, com a qual temos um pouco de experiência, para realizar o desenvolvimento deste jogo.

	O jogo será divido em 3 atos, sendo cada um deles subdividido em diversas fases, com segmentos de texto que contam
uma história entre elas. O objetivo de cada fase é chegar num local específico, que ainda será definido (como uma porta,
por exemplo).

	O primeiro ato servirá para introduzir as mecânicas do jogo para o jogador. A tela será dividida em duas, sendo
cada lado uma reflexão perfeita do outro. O jogador controlará então seu personagem e sua reflexão de forma 
paralela.

	O segundo ato ainda terá a tela dividida, mas a reflexão não será mais perfeita. Isso pode levar a dessincronização
entre o personagem e seu reflexo, ainda controlado de forma paralela, o que proporcionará desafios diferentes
dos anteriores.

	No terceiro e último ato, a tela deixará de ser dividida, e o esquema de controle do personagem será sequencial,
através de clones que repetirão os movimentos anteriores do personagem principal.   

\subsection{Objetivo final}
	Nosso objetivo é ter um jogo completo até o fim do semestre, disponível para \textit{download} no site
da USPGameDev. Também pretendemos apresentar o jogo na \textit{let's test play} que será realizado no início
do semestre seguinte. 


\section{Etapas de desenvolvimento}

	\subsection{Desenvolvimento das mecânicas básicas} - Movimento do personagem, pulo, implementar as reflexões e interação com objetos (como botões, alavancas, empurrar caixas, etc). Também implementar o menu principal do jogo.

	\subsection{Planejamento, implementação e teste das fases} - Uso das mecânicas básicas em conjunto com \textit{assets placeholder} para criar as fases do jogo, assim como definir um esqueleto da história.

	\subsection{Criação dos \textit{assets} finais} - Criação da arte, animação, música e efeitos sonoros que serão utilizados no jogo, assim como esquematização e impressão do pôster.

	\subsection{Testes} - Correção de \textit{bugs}, \textit{playtesting} com mais pessoas, finalização do projeto e escrita do relatório final.

\section{Cronograma}
		Faremos 2 reuniões por semana, cada uma com duração de 3h. Elas poderão ser presenciais, no
Laboratório 16 do CCSL, ou feitas remotamente.
  	
  	\subsection{Dia e horário das reuniões}
  	\begin{itemize} 
  	\item Segunda-feira - 9:00 às 12:00
  	\item Quarta-feira - 13:00 às 16:00
 	\end{itemize}

 	Totalizando 96h por pessoa, sendo as 4h adicionais completadas durante as semanas de \textit{BREAK}.

\subsection{Planejamento semanal}

\begin{itemize} 
\item Semana 1 (12/08 à 18/08)
 \begin{itemize} 
 \item Aprender a mexer na versão 3.0 da \textit{Godot}
 \item Definir as mecânicas básicas e começar a implementá-las
  \end{itemize}

\item Semana 2 (19/08 à 25/08)
 \begin{itemize} 
 \item Continuar a implementação das mecânicas básicas
 \item Começar a fazer os \textit{assets placeholders}
 \end{itemize}

\item Semana 3 (26/08 à 01/09)
 \begin{itemize} 
 \item Continuar a implementação das mecânicas básicas
 \item Terminar os \textit{assets placeholders}
 \end{itemize}

\item Semana 4 (02/09 à 08/09) (\textit{BREAK})
 \begin{itemize} 
 \item Continuar a implementação das mecânicas básicas
 \item Começar a implementação das fases iniciais (tutoriais) e menus
 \end{itemize}

\item Semana 5 (09/09 à 15/09)
 \begin{itemize} 
 \item Terminar a implementação das mecânicas básicas
 \item Terminar as fases iniciais e menus
 \end{itemize}

\item Semana 6 (16/09 à 22/09)
 \begin{itemize} 
 \item Planejamento, teste e implementação das fases do Ato 1
 \item Fazer o esqueleto da história do jogo
 \end{itemize}

\item Semana 7 (23/09 à 29/09)
 \begin{itemize} 
 \item Planejamento, teste e implementação das fases do Ato 1
 \item Escrever a história do jogo
 \end{itemize}

\item Semana 8 (30/09 à 06/10)
 \begin{itemize} 
 \item Planejamento, teste e implementação das fases do Ato 2
 \item Escrever a história do jogo
 \end{itemize}

\item Semana 9 (07/10 à 13/10) (\textit{BREAK})
 \begin{itemize} 
 \item Planejamento, teste e implementação das fases do Ato 2 
 \item Listar os \textit{assets} finais do jogo (gráficos, animações e música)
 \end{itemize}

\item Semana 10 (14/10 à 20/10)
 \begin{itemize} 
 \item Planejamento, teste e implementação das fases do Ato 2
 \item Fazer os \textit{assets} finais do jogo
 \end{itemize}

\item Semana 11 (21/10 à 27/10) 
 \begin{itemize} 
 \item Planejamento, teste e implementação das fases do Ato 3
 \item Fazer os \textit{assets} finais do jogo
 \end{itemize}

\item Semana 12 (28/11 à 03/11)
 \begin{itemize} 
 \item Planejamento, teste e implementação das fases do Ato 3
 \item Fazer os \textit{assets} finais do jogo
 \end{itemize}

\item Semana 13 (04/11 à 10/11) 
 \begin{itemize} 
 \item Fazer os \textit{assets} finais do jogo 
 \item Imprimir o pôster
 \end{itemize}

\item Semana 14 (11/11 à 17/11) (\textit{BREAK})
 \begin{itemize} 
 \item Pôster - entrega
 \item Fazer os \textit{assets} finais do jogo
 \end{itemize}

\item Semana 15 (18 à 24/11)
 \begin{itemize} 
 \item Correção de \textit{bugs} e \textit{playtesting}
 \item Escrever o relatório final
 \end{itemize}

\item Semana 16 (25/11 à 01/12)
 \begin{itemize} 
 \item Correção de \textit{bugs} e \textit{playtesting}
 \item Escrever o relatório final
 \end{itemize}

\end{itemize}

\section{Acompanhamento}
 	O progresso do projeto poderá ser acompanhado no nosso repositório do GitHub \url{https://github.com/papukweh/Sohlepse}, que também possuirá uma wiki com detalhes do que foi feito em cada semana.

 \subsection{Orientador} Nosso orientador será Wilson Kazuo Mizutani (\href{mailto:kazuo@ime.usp.br}{kazuo@ime.usp.br}), membro ativo da USPGameDev.

\end{document}
