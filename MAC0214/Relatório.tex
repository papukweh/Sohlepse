\documentclass[a4paper, 11pt]{article} 
\usepackage[utf8]{inputenc} 
\usepackage{indentfirst} 
 
\title{\textbf{Sohlepse} \\ \small{Um Game Design Document}} 
\author{Larissa Goto Sala, Pedro Paulo de Queiroz Bambace, Nathalia Orlandi Borin} 
 
\begin{document} 
 
\maketitle 
 
\newpage 
 
\newpage 
 
\section{Resumo} 
	 
	Sohlepse é um jogo de puzzle-plataforma que faz uso de reflexões para proporcionar o controle de mais de um personagem com os mesmos comandos. 
 
Isso pode ser feito de forma paralela (os dois personagens se movem ao mesmo tempo) ou sequencial (a mesma sequência de movimentos de um personagem é repetida enquanto você move o outro personagem). 
 
Cada fase apresentará diferentes obstáculos que te obrigará a fazer uso criativo de suas reflexões para superá-los. 
 
\subsection{Histórico}

	\subsubsection{Semana 1 (12/08 à 18/08)}
		\begin{itemize} 
	 		\item \textbf{Reunião do dia 13/08/2018}

	 		\begin{itemize}
	   			\item Estudamos e alteramos os códigos do tutorial de platformer 2d da Godot.
	    		\item Invertemos o movimento do personagem e tentamos fazer o sistema de puxar/empurrar caixas (que ainda possui problemas: a caixa não desliza da maneira que queríamos).
	    	\end{itemize}

			\item \textbf{Reunião do dia 15/08/2018}

			\begin{itemize}
	    		\item Temos um personagem com atributos que definem se o movimento é espelhado horizontalmente, verticalmente ou ambos. Ele também consegue puxar e empurrar caixas.
	    		\item Temos botões que funcionam quando pressionados e alavancas com as quais o jogador pode interagir.
	    		\item Temos uma tentativa de implementação de um chão quebrável (há alguns problemas com a função free()).
	    	\end{itemize}

	    	\item \textbf{Total de horas trabalhadas:}
	    		\begin{itemize}
	    			\item Larissa:
	    			\item Nathalia:
	    			\item Pedro Paulo:
	    		\end{itemize}

	  	\end{itemize}

\subsubsection{Semana 2 (19/08 à 25/08)}
	\begin{itemize} 
		\item \textbf{Reunião do dia 20/08/2018}

		\begin{itemize}
	    	\item A física do movimento do personagem foi melhorada.
	    	\item Consertamos alguns bugs de colisão com a caixa.
	    	\item Temos um tileset provisório.
	    	\item Temos um endgoal e conseguimos mudar de fase.
	    	\item Tentamos arrumar a colisão dos chãos quebráveis.
	    \end{itemize}

		\item \textbf{Reunião do dia 22/08/2018}

		\begin{itemize}
	    	\item Temos duas telas com câmeras que seguem o jogador e seu clone.
	    	\item O jogador agora pode morrer por queda ou se encostar em tiles especiais, como lava.
	    	\item Temos um tileset melhorado e assets para o player, endgoal, caixa, lava, água e paredes.
	 	\end{itemize}

	\end{itemize}

\subsubsection{Semana 3 (26/08 à 01/09)}
	\begin{itemize} 
		\item \textbf{Reunião do dia 27/08/2018}

		\begin{itemize}
    		\item Temos um menu de seleção de fases desbloqueáveis e um arquivo que salva (permanentemente) o progresso do jogador.
    		\item Temos assets para botões e alavancas.
    		\item Nosso endgoal foi alterado para suportar mais de um jogador.
    		\item O jogador agora tem sua velocidade diminuída dependendo do terreno onde se encontra (fica mais lento na água, por exemplo).
    		\item Temos um menu principal com as opções de continuar, selecionar fase, opções, créditos e sair do jogo.
   			\item Aumentamos a resolução do jogo.
   		\end{itemize}

		\item \textbf{Reunião do dia 29/08/2018}

		\begin{itemize}
    		\item Temos menus mais bonitos, com botões de voltar e uso do teclado para selecionar as opções.
    		\item Nosso tile de água agora funciona melhor para tiles adjacentes.
    		\item O Stage Manager agora pode receber informação da quantidade de câmeras/repartições necessárias em cada fase.
    		\item Mecânica dos clones parcialmente implementada.

		\end{itemize}

	\end{itemize}

\subsubsection{Semana 4 (02/09 à 08/09) (\textit{BREAK})}
	\begin{itemize} 
		\item \textbf{Reunião do dia 03/09/2018}

		\begin{itemize}
    		
    		\item Sistema de clones finalizado.
    		\item Arquivos foram organizados em diretórios.
    		\item Confirmação para sair no menu in-game, bug fixes nos menus.
    		\item Temos plataformas de pulo (one-way) e plataformas que se movem.
    		\item Botões e alavanca agora servem para objetos genéricos.
    		\item Espinhos acionáveis implementados.
    		\item Começamos a fazer as fases iniciais de tutorial do jogo.

   		\end{itemize}

		\item \textbf{Reunião do dia 05/09/2018}

		\begin{itemize}
    		
    		\item Caixa não fica mais presa em certos lugares.
    		\item Sistema de clones arrumado (número máximo de clones, interação deles com objetos e reset completo da fase ao criá-los).
    		\item Física do player em cima da plataforma móvel consertada.
    		\item Sprite do player e sua animação de morte agora estão espelhadas.
    		\item Temos um menu para checagem dos controles do jogo dentro do menu de pause.
    		\item Continuação da criação de fases tutoriais.

		\end{itemize}

	\end{itemize}


\subsubsection{Semana 5 (09/09 à 15/09)}
	\begin{itemize} 
		\item \textbf{Reunião do dia 10/09/2018}

		\begin{itemize}
    		
    		\item Duas novas fases estão em progresso.
    		\item Player agora pode morrer esmagado.
    		\item Tile adicional de chão adicionado.
    		\item Consertado bug com espelho do player.
    		\item Alavancas e botões agora emitem sinais para grupos de Nodes.

   		\end{itemize}

		\item \textbf{Reunião do dia 12/09/2018}

		\begin{itemize}
    			
    		\item Agora há um separador entre o lado real e o espelho.
    		\item Colocamos um raycast em cada pé do player para que ele conseguisse pular estando nas pontas dos tiles.
    		\item Pulo em plataforma one-way arrumado.
    		\item Modo diferente de ativar botões e alavancas.
    		\item Um objeto pode ser acionado por dois botões diferentes.
    		\item Mais uma fase tutorial pronta.

		\end{itemize}

	\end{itemize}

\subsubsection{Semana 6 (16/09 à 22/09)}
	\begin{itemize} 
		\item \textbf{Reunião do dia 10/09/2018}

		\begin{itemize}
    		
    		\item Mais uma fase em progresso.
    		\item Correção de bugs (ou tentativa de): botões que precisam ficar pressionados, caixas que não são empurradas quando deveriam, colisões bugadas.
    		\item Mudança no sistema de pulo - agora é um action_just_pressed()
    		\item Edição de fases anteriores para ficarem compatíveis com os novos botões.

   		\end{itemize}

		\item \textbf{Reunião do dia 12/09/2018}

		\begin{itemize}
    					
    		\item Brainstorm para a possível história.
    		\item GDD atualizado.
    		\item Tentativa de correção de bugs: dessincronização do player e caixas travadas.
    		\item Água arrumada com o novo sistema de pulo.
    		\item Plataformas que se movem com possibilidade de pulo de baixo para cima (one-way).
    		\item Plataformas estáticas agora evitam pulo duplo.
    		\item Início da modularização dos labels nos tutoriais.
    		\item Começamos a mudar o tutorial para primeira pessoa.

		\end{itemize}

	\end{itemize}

\subsubsection{Semana 7 (23/09 à 29/09)}
	\begin{itemize} 
		\item \textbf{Reunião do dia 24/09/2018}

		\begin{itemize}
    		
    		\item Definimos os acontecimentos principais da história assim como o cenário, natureza das "cutscenes" e estrutura de atos.
    		\item Tentativa de separar fases teste de fases definitivas (pode ou não ter quebrado tudo).
    		\item Menu ainda mais modularizado.
    		\item Tentativa de arrumar plataformas que se movem one way.
    		\item Início de criação de 3 novas fases.

   		\end{itemize}

		\item \textbf{Reunião do dia 26/09/2018}

		\begin{itemize}
    					
    		\item Definimos a ordem e quantidade de fases.
    		\item Listamos os assets finais.
    		\item Definimos melhor as cutscenes do jogo.
    		\item Testamos a possibilidade de portar nosso jogo para Godot 3.1 Alpha (conserta o problema das plataformas que se movem mas quebra outras coisas).

		\end{itemize}

	\end{itemize}

\subsubsection{Semana 8 (30/09 à 06/10)}
	\begin{itemize} 
		\item \textbf{Reunião do dia 01/10/2018}

		\begin{itemize}
    		
    		\item Duas fases do Ato 1 concluídas - uma de tutorial, e outra mais difícil.
    		\item Tentamos corrigir as dessincronizações do ato 1 mudando a forma como o jogo é espelhado, não tivemos sucesso até então.
    		\item Bugs de dessincronização na fase de água e lava corrigidos.
    		\item Sistema de tutoriais agora tem suporte a sprites e tamanho de áreas diferentes.

   		\end{itemize}

		\item \textbf{Reunião do dia 03/10/2018}

		\begin{itemize}
    					
    		\item Uma fase do Ato 1 e uma fase do Ato 3 concluídas.
    		\item Nova forma de espelhar as fases do Ato 1 quase completa.
    		\item Agora temos dois tipos de falling floor.
    		\item Tutoriais reescritos em primeira pessoa.

		\end{itemize}

	\end{itemize}

\subsubsection{Semana 9 (07/10 à 13/10) (\textit{BREAK})}
	\begin{itemize} 
		\item \textbf{Reunião do dia 08/10/2018}

		\begin{itemize}
    		
    		\item Duas fases do Ato 1 e duas fases do Ato 2 concluídas.
    		\item Fix no menu de fases, suporte à mais de 12 fases.
    		\item Endgoals agora são afetados pela gravidade.

   		\end{itemize}

		\item \textbf{Reunião do dia 10/10/2018}

		\begin{itemize}

    		\item Tentamos novamente portar o jogo para Godot Alpha 3.1, mas poucos bugs foram corrigidos.
    		\item Plataformas one-way podem ter movimentos verticais agora.
    		\item Plataformas normais estão flickando menos.
    		\item Correção de bugs de morte por esmagamento: tempo de morte menor e plataformas one-way não matam mais.
    		\item Fases espelhadas horizontalmente agora possuem outro método para espelhar (invertendo o viewport ao invés de inverter cada objeto do mapa)
    		\item Duas novas fases do Ato 2.
    		\item Espinhos agora matam o player se forem acionados com o player sobre o tile.

		\end{itemize}

	\end{itemize}

\subsubsection{Semana 10 (14/10 à 20/10)}
	\begin{itemize} 
		\item \textbf{Reunião do dia 15/10/2018}

		\begin{itemize}

   			\item Sprites diferenciados para paredes ativadas por mais de um botão e botões de pressão/instantâneos.
    		\item Fases do Ato 1 espelhadas de verdade usando zoom negativo da câmera, mantendo os limites originais.
    		\item Código de esmagamento do player e das plataformas melhorado, mas ainda há bugs.
    		\item Protótipo de fase do ato 2 concluído.
    		\item Correção de bug relativo aos espinhos na presença de caixas.

   		\end{itemize}

		\item \textbf{Reunião do dia 17/10/2018}

		\begin{itemize}

    		\item Códigos de empurrar caixas e de colisão do player com caixas e plataformas melhorados.
    		\item Fase final do Ato 1 começou a ser implementada.
    		\item Nova fase do Ato 2.
    		\item Modo cumulativo de acionar botões/alavancas agora funciona quando eles ativam múltiplas coisas.

		\end{itemize}

	\end{itemize}

\subsubsection{Semana 11 (21/10 à 27/10)}
	\begin{itemize} 
		\item \textbf{Reunião do dia 22/10/2018}

		\begin{itemize}

    		\item Novo tile de background para levers.
    		\item Nova cena BigWall para evitar distorções em walls esticadas.
    		\item Fase final do Ato 1 completa.
    		\item Nova fase do Ato 3.
    		\item Música tema do laboratório (Ato 3?) finalizada.

   		\end{itemize}

		\item \textbf{Reunião do dia 24/10/2018}

		\begin{itemize}

    		\item Assets definitivos do laboratório em progresso: mais opções de tilesets e fixes estéticos.
    		\item Duas novas fases do Ato 2 em progresso, e uma do Ato 3.
    		\item Consertado bug de colisão com tileset e cabecear caixas.
    		\item Agora as fases do Ato 2 estão no menu de fases oficial, na ordem devida.
    		\item Playtesting com pessoas que não fazem parte do desenvolvimento.

		\end{itemize}

	\end{itemize}

\subsubsection{Semana 12 (28/10 à 03/11)}
	\begin{itemize} 
		\item \textbf{Reunião do dia 29/10/2018}

		\begin{itemize}

    		\item Mais assets definitivos do laboratório concluídos.
    		\item Duas novas fases do Ato 2 e duas fases do Ato 3.
    		\item Novas mecânicas do ato 3: indicação de gravação, número de clones permitidos e uso de cadáver de clones como "chão".
    		\item Temos agora MovingThorns.

   		\end{itemize}

		\item \textbf{Reunião do dia 31/10/2018}

		\begin{itemize}

    		\item Consertado bug do duplo pulo e de alavancas crashando o jogo.
    		\item Temos um player definitivo e com animações.
    		\item Início de nova fase do ato 3.
    		\item Mais playtesting.

		\end{itemize}

	\end{itemize}

\subsubsection{Semana 13 (04/11 à 10/11)}
	\begin{itemize} 
		\item \textbf{Reunião do dia 05/11/2018}

		\begin{itemize}

    		\item Fizemos o pôster com imagens placeholders.
    		\item Começamos a fazer o background em parallax.
    		\item O cadáver do player agora possui física: seus clones mortos podem ser usados como "pontes" sobre tiles perigosos como espetos e lava (Ato 3).
    		\item Mais playtesting.

   		\end{itemize}

		\item \textbf{Reunião do dia 07/11/2018}

		\begin{itemize}

    		\item Pôster finalizado, impresso e afixado no mural.
    		\item Backgrounds estáticos e novo sprite para alavancas.
    		\item Protótipo para mais 4 fases: duas do Ato 2 e duas do Ato 3.
    		\item Duas novas músicas: das fases de Ato 2 e do Menu principal.

		\end{itemize}

	\end{itemize}

\subsubsection{Semana 14 (11/11 à 17/11) (\textit{BREAK})}
	\begin{itemize} 
		\item \textbf{Reunião do dia 12/11/2018}

		\begin{itemize}

    		\item Implementamos 4 fases prototipadas na reunião passada.
    		\item Começamos o design dos assets para a floresta (Ato 2).
    		\item Iniciamos a implementação de um sistema de som.
    		\item Músicas do Menu, Ato 1 e Ato 2 agora estão no jogo.
    		\item Começamos a caçar efeitos sonoros no freesound.org.

   		\end{itemize}

		\item \textbf{Reunião do dia 14/11/2018}

		\begin{itemize}

    		\item Assets para a floresta finalizados.
    		\item Indicação de que o player não pode gravar em sua posição inicial (Ato 3).
    		\item Todos os efeitos sonoros dos Atos 1 e 3 implementados.
    		\item Menu de Options agora tem controladores de volume e mostra os controles (teclado apenas).
    		\item Colocamos as 30 fases do jogo no Stage Select oficial.

		\end{itemize}

	\end{itemize}

\subsubsection{Semana 15 (18/11 à 24/11)}
	\begin{itemize} 
		\item \textbf{Reunião do dia 19/11/2018}

		\begin{itemize}

   			\item  Menus definitivos finalizados.
    		\item Fase final do Ato 2 concluída.
    		\item Fase final do Ato 3 em progresso.
    		\item Funcionamento do MovingThorns modificado (agora desligar a plataforma também desliga os espinhos).

   		\end{itemize}

		\item \textbf{Reunião do dia 21/11/2018}

		\begin{itemize}

    		\item Fizemos a cutscene do Ato 1 para o Ato 2.
    		\item Mostramos o jogo na reunião de integração.
    		\item Começamos a implementar o timer da fase final.
    		\item Começamos a colocar o menu de options in-game.

		\end{itemize}

	\end{itemize}

\subsubsection{Semana 15 (25/11 à 01/12)}
	\begin{itemize} 
		\item \textbf{Reunião do dia 26/11/2018}

		\begin{itemize}

    		\item Fizemos a cutscene do Ato 2 para o Ato 3.
    		\item Timer da fase final implementado.
    		\item Background parallaxe de quase todas as fases implementado.
    		\item Menu options in-game funcional.
    		\item Correção de bugs variados.

   		\end{itemize}

		\item \textbf{Reunião do dia 28/11/2018}

		\begin{itemize}

    		\item Escrevemos este relatório.

		\end{itemize}

	\end{itemize}


\section{Resultado Final} 
 
	\subsection{Ações do jogador} 
		\begin{itemize} 
			\item \textbf{Andar:} $\rightarrow$ e $\leftarrow$ ou $A$ e $D$  
			\item \textbf{Pular:} $\uparrow$ ou $W$
			\item \textbf{Empurrar caixas:} $E$ enquanto anda na direção da caixa  
			\item \textbf{Pressionar botões:} permanecer em cima dele 
			\item \textbf{Acionar alavancas:} $E$
		\end{itemize} 
 
	\subsection{Objetos do jogo} 
		\begin{itemize} 
			\item \textbf{Botões:} ativam ou desativam uma quantidade arbitrária de paredes e espinhos divididos em 2 tipos: 
			\begin{itemize}
			        \item Os que se mantêm ativos enquanto estão pressionados
			        \item Os que pressionados uma vez, se mantém ativos até o fim da fase
			\end{itemize}   
			\item \textbf{Alavancas} - Se mantêm ativos até serem acionadas novamente - ativam ou desativam uma quantidade arbitrária de paredes e espinhos 
			\item \textbf{Caixas} - Podem ser empurradas para formar escadas ou acionar botões 
			\item \textbf{Chão quebrável} - Se quebram quando o jogador passa por ele duas vezes 
			\item \textbf{Água} - Jogador pode passar por esse tile normalmente, mas sua movimentação fica mais lenta 
			\item \textbf{Lava} - Jogador morre se encostar nesse tile 
			\item \textbf{Espetos} - Jogador morre se encostar nesse tile - pode ser desativado por um botão/alavanca
			\item \textbf{MovingEspetos} - Como um tile de espeto, mas que se move 
		\end{itemize} 
 		\subsection{EndGoal}
 			Espelho que representa o ponto onde o Jogador deve chegar para terminar a fase, ao interagir com ele, a próxima fase começa(dependendo da fase podem haver cutscenes antes da inicialização)
	
\section{Ato I} 
 
	Duas telas espelhadas, podendo a divisão ser na horizontal ou vertical, dependendo da fase. O movimento do 
	personagem na primeira tela é refletido de maneira idêntica na segunda tela. Uma das telas pode ser obscurecida 
	e/ou conter partes quebradas/cobertas dependendo da fase. 
	Nesse ato, o principal objetivo é introduzir o jogador às mecânicas do jogo. As fases são idênticas 
	em ambos lados do espelho (salvo possíveis exceções onde o espelho será obscurecido/quebrado). 
 
 
\subsection{Fases} 
 
	\begin{itemize} 
		\item \textbf{Fase1} : Tutorial de movimento, interação e pausa 
		\item \textbf{Fase2} : Introduz caixas para subir e alavancas
		\item \textbf{Fase3} : Introduz água, lava e espinhos
		\item \textbf{Fase4} : Introduz caixas para apertar botões
		\item \textbf{Fase5} : Introduz espinhos acionáveis e plataformas one-way
		\item \textbf{Fase6} : Introduz falling floor e plataformas que se movem
		\item \textbf{Fase7} : Quebrar falling floors de propósito é uma estratégia válida, introduz alavancas cumulativas
		\item \textbf{Fase8} : Plataformas e caixas: não fazer o óbvio primeiro (tacar no buraco)
		\item \textbf{Fase9} : Tutorial de restart, não fazer o óbvio primeiro (novamente)
		\item \textbf{Fase10} : Introduz mecânica de plataforma que empurra caixa, morte por esmagamento
		\item \textbf{Fase11} : Uso de falling floors, alavancas que ativam mais de uma coisa - patience platforming (pequenas diferenças entre água e lava)
		\item \textbf{Fase12} : Fase final do Ato 1 - plataformas ativando botões, traps com espinhos, uso de dessincronização com paredes acionáveis, falling floor com timing preciso.
	\end{itemize} 
 
\subsection{História} 
 
	Jogador acorda em uma sala fechada, com apenas um espelho, sem saber onde está. Ao interagir com o espelho, é puxado pela própria reflexão para o "mundo espelhado".
	Percorre o laboratório e vai percebendo sutis mudanças na reflexão das fases. Eventualmente, encontra uma saída, e acaba perdido numa floresta desconhecida. Termina o ato jurando que voltará para sua família.

\section{Ato II} 
 
	Duas telas que representam fases diferentes, podendo a divisão ser na horizontal ou vertical, dependendo da fase. 
	O movimento do personagem na primeira tela é refletido na segunda tela, mas como os cenários são diferentes, pode 
	haver uma dessincronização entre os movimentos 
	Nesse ato, o mais longo do jogo, o objetivo é brincar com a noção de realidades paralelas, onde você deve planejar 
	bem seus movimentos para controlar dois personagens em cenários distintos com os mesmos inputs. Ao longo das fases, 
	a barreira entre as duas realidades começa a se quebrar, e você pode acabar interagindo com sua "reflexão" de  
	maneiras diferentes. 
 
\subsection{Fases} 
 
	\begin{itemize} 
		\item \textbf{Fase13} : Introduz mecânica de alavancas que ativam mais de uma coisa. O puzzle é descobrir a ordem certa de ativar as alavancas.
		\item \textbf{Fase14} : Clone fica ativando/desativando plataformas para o player - uso da caixa para sincronizar os movimentos sem que o player caia no espinho.
		\item \textbf{Fase15} : Clone fica ativando/desativando plataformas para o player, tomando cuidado para não ser esmagado e/ou desativar o próprio chão.
		\item \textbf{Fase16} : Traps envolvendo espinhos e caixas, depende de um "leap of faith" do player e seu clone, que desativa os espinhos da fase antes do player cair neles.
		\item \textbf{Fase17} : Uso de falling floors no lado espelhado, de forma que os movimentos do clone não pode ser 1:1 os do player. O segredo da fase está na sincronização do movimento dos dois no pulo da terceira plataforma para a quarta (o clone deve aproveitar esse momento para chegar na alavanca e ativá-la). Introduz mecânica de espelho que se move.
		\item \textbf{Fase18} : Fase "Mario Maker" - jogador não pode pular e nem ir muito para os lados, ou o clone morre. A ideia é sincronizar as plataformas que se movem de antemão, e depois só seguir o fluxo.
		\item \textbf{Fase19} : Fase em que o player deve ativar diversos botões ao mesmo tempo com seu clone, precisando usar da dessincronização para isso.
		\item \textbf{Fase20} : Fase em que o player precisa usar um tile bem colocado para dessincronizar seu movimento em relação a uma plataforma que se move.
		\item \textbf{Fase21} : Fase "Mario Maker Invertido" - jogador só pode pular em uma pequena área na fase e seu clone deve ativar e desativar plataformas de forma a levar o espelho até ele. É preciso tomar cuidado para não mandar o espelho pro abismo.
		\item \textbf{Fase22} : Plataformas "troll" e traps com espinhos. Uso de uma parede acionável para dessincronizar os jogadores (callback para \textbf{fase12}). Uso de informação no lado espelhado para guiar ações do player no lado real (o movimento da plataforma que carrega o espelho está sincronizado com a plataforma espelhada).
		\item \textbf{Fase23} : Fase de corrida - o player deve desativar as paredes e derrubar as caixas antes que elas sejam levadas por outra plataforma - essa é a única maneira de atravessar a lava. Introduz meĉanica de espinho que se move.
		\item \textbf{Fase24} : Fase final do Ato 2 - espinhos que se movem, dessincronização usando paredes. espiral da morte com espinhos, timear bem o pulo dos dois players.
	\end{itemize} 
 
 
\subsection{História} 
 
	O jogador percebe que sua reflexão agora tem olhos vermelhos. Ao longo das fases, sente-se mal e começa a esquecer acontecimentos importantes em sua própria vida. Coopera com sua reflexão contra sua vontade, tendo sempre um pouco de desconfiança com relação à esse. Na fase final, a realidade espelhada e a real se juntam numa só, e o player finalmente chega em um lugar que parece sua casa. Sua reflexão o aconselha a não entrar. O jogador a ignora e entra em sua própria casa.
 
\section{Ato III} 
 
	Nesse ato, a realidade do espelho se fundiu quase completamente com a realidade, então as fases se passam numa única 
	tela. O diferencial desse ato é o uso de "clones", cujo número pode ser limitado ou não, dependendo da fase, que repetem o que você faz, como numa gravação. A ideia é sincronizar o seu movimento com o dos seus "clones" para conseguir atingir seu objetivo. 
 
\subsection{Fases} 
 
	\begin{itemize} 
		\item \textbf{Fase25} : Callback da primeira fase: mas dessa vez o espelho está no "teto". Introduz mecânica de pular nos clones para atingir seu objetivo.
		\item \textbf{Fase26} : Callback da primeira fase do ato 2: dessa vez quem ativa as plataformas é um clone gravado, e não um clone do espelho.
		\item \textbf{Fase27} : Player tem acesso a apenas um clone e deve ativar todos os botões para abrir a parede que protege o espelho.
		\item \textbf{Fase28} : Introduz mecânica de matar clones para prosseguir - player deve usar uma plataforma com espinhos para atingir seu objetivo, e para isso precisa ficar sobre o cadáver de um clone.
		\item \textbf{Fase29} : Como na fase anterior, temos apenas um clone, que deve percorrer a fase e ativar um botão que abre a parede que protege o espelho. Porém, o sucesso da gravação depende do player ativando/desativando plataformas e paredes no momento certo.
		\item \textbf{Fase30} : Mostra a importância de começar sua gravação no momento certo - é possível fechar o espelho apertando um botão muito bem posicionado. A gravação deve ativar e desativar espinhos/paredes em um ritmo no qual o jogador possa passar por eles com facilidade.
		\item \textbf{Fase31} : Nesta fase é preciso timear bem suas gravações: um movimento que ativava uma alavanca inicialmente pode ser reutilizado para ativar outra, se você usar bem seus recursos. Introduz mecânica de carregar um clone em sua cabeça.
		\item \textbf{Fase32} : Fase final - corrida contra o tempo. Essa fase apresenta um timer e está repleta de armadilhas. O player deve ter pressa e timear muito bem seus pulos para atingir o final.
	\end{itemize} 

\subsection{História} 
 
	O jogador novamente acaba na sala inicial do laboratório. Exasperado, recomeça sua busca por uma saída. Quando finalmente a encontra, seu clone aparece e lhe diz que apenas um deles pode sair daquele lugar vivo, e ativa um mecanismo de auto-destruição do laboratório. O jogador declara que não deixará que o clone roube sua identidade, e começa então uma corrida para a saída. Se o jogador conseguir escapar a tempo, ele chega em casa e o clone acaba morrendo na explosão. Em casa, entretanto, depara-se com uma surpresa - sua esposa tem olhos vermelhos, como o seu clone. 

\subsection{Considerações finais}
	O que fizemos, o que faltou fazer, dificuldades, o que aprendemos
 	Inserir imagens das coisas (as do pôster?) also menu bonito
 	Colocar horas?
 	Referências para ferramentas?
 
\end{document} 
 